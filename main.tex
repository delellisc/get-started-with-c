\documentclass{article}

% Language setting
% Replace `english' with e.g. `spanish' to change the document language
\usepackage[english]{babel}

% Set page size and margins
% Replace `letterpaper' with `a4paper' for UK/EU standard size
\usepackage[letterpaper,top=2cm,bottom=2cm,left=3cm,right=3cm,marginparwidth=1.75cm]{geometry}

% Useful packages
\usepackage{amsmath}
\usepackage{graphicx}
\usepackage[colorlinks=true, allcolors=blue]{hyperref}

% quotes
\usepackage{dirtytalk}

% code blocks
\usepackage[outputdir=build, cachedir=build/_minted-notes]{minted}

\title{Beginner at C/C++}
\author{Camilo de Lellis}

\begin{document}
\maketitle
\tableofcontents

\section{Introduction to C/C++}

% -----|-----|-----|-----|-----|-----|-----|-----|-----|-----

\subsection{What is C++}
C++ is a \textit{object-oriented programming language} being used in the most diverse applications. For this reason it is called a \textit{general purpose language} \cite{wikibooks_cpp}. It is a \textit{superset} of the C programming language, being a extension including object oriented features that C lacks.

C++ was originally called "C with classes" and was concepted when Bjarne Stroustrup, a computer scientist, thought that it would be cool to add object-oriented concepts to C. He then added classes, virtual functinos, operator overloading, multiple inheritance, templates and exception handling. This was how C++ was born. \cite{wikibooks_cpp}

Some other features introduced by C++ were:

\say{[...] declarations as statements, function-like casts, new/delete, bool, reference types, const, inline functions, default arguments, function overloading, namespaces, classes (including all class-related features such as inheritance, member functions, virtual functions, abstract classes, and constructors), operator overloading, templates, the :: operator, exception handling, run-time type identification, and more type checking in several cases}\cite{wikibooks_cpp}

% -----|-----|-----|-----|-----|-----|-----|-----|-----|-----

\subsection{Why use C/C++}

After some thought, I figured that I wanted to work with ML hardware, algorithm optimization, GPU development and signal processing. Such areas need a strong understanding of math, programming and computer architecture. In that sense, it was indispensable to learn a general purpose language that would allow me to have access to the hardware and software as much as possible. C and C++ came to mind and were pretty much confirmed as my ideal targets when I came to know that they were used a lot in this area that I became fond of. 

With this repo, I intend to get strong foundations in the fundamentals of C and C++, even though I am a bit familiar with it's syntax, since most of my projects were focused on web development, I am quite unfamiliar with how these languages can be used in the context I intend to apply it.I intend to write these notes with C++ in mind because it seems the more suitable from a market perspective, but I want to compare the code to C as much as possible and say where it would be better to implement something in C and when it would be better to do so in C++.

That should make we ask ourselves, why do people still use C if C++ is supposed to do all C does and much more? The answer is fairly simple. Let's take the operating systems called Linux as an example. Its kernel is written in C, even though C++ has many more features. In the case of a operating system kernel, performance is everything, and those feature could be quite detrimental. Object oriented code with classes and virtual lookup tables for method overloading add some steps that do not really offer any value. C is much closer to ASM than C++, making it perfect for the job. It is also safe to say that C is incredibly much simpler, that makes it really simple to learn and debugg, different from C++. \cite{CaptainAwesomePants2021} In that sense, when features missing from C are needed for a project, the wise choice is to pick C++.

% -----|-----|-----|-----|-----|-----|-----|-----|-----|-----

\subsection{Setting up a text editor}
To write code in C and C++ we need to adopt a text editor. A text editor is nothing more than a computer program used to write inside files.

The first editor we'll show how to set up is Visual Studio Code from Microsoft. This is the \href{https://code.visualstudio.com/download}{download link} for VSCode. For this repo, I'll adopt Linux, Ubuntu 24.04.3 LTS as the default operating system. To download the \href{https://go.microsoft.com/fwlink/?LinkID=760868}{latest version} at time of writing, you can use the following command:

\begin{minted}[breaklines, breakanywhere]{bash}
wget https://vscode.download.prss.microsoft.com/dbazure/download/stable/7d842fb85a0275a4a8e4d7e040d2625abbf7f084/code_1.105.1-1760482543_amd64.deb
\end{minted}

After that, you just need to install it using the command:

\begin{minted}{bash}
sudo dpkg -i code_1.105.1-1760482543_amd64.deb
sudo apt-get install -f
\end{minted}

When doing so, you will be prompted to install the apt repository. I believe the fastest way to install the necessary extensions is to create a C/C++ file and open it using VSCode. You can just create a new project and create a file using the GUI. I like to do it via terminal. So, for example, if I wanted to open a folder and create a file inside it, I would do something like this:
\begin{minted}{bash}
# Assuming that this is the target folder
cd ~/projects/c-cpp/get-started-with-c
touch sample.c
\end{minted}

If you click to open the file just created, the pop-up in Figure 1 will appear. Just click install and you will be set up with the default C/C++ development configuration for VSCode.
\begin{figure}
\centering
\includegraphics[width=0.25\linewidth]{./images/c-cpp-extension-popup.png}
\caption{\label{fig:extension-popup}Pop up that appears when you open a C/C++ file in VSCode.}
\end{figure}

% -----|-----|-----|-----|-----|-----|-----|-----|-----|-----

\subsection{Running your first program}
Running C code is a fairly simple process. We shall now see the steps needed for doing so. First, we create a file:
\begin{minted}{bash}
touch hello-world.c
\end{minted}

Write the code inside it:
\begin{minted}{c}
#include <stdio.h>

int main( ) {
    printf("hello, world");
    return 0;
}
\end{minted}

Run it via command line:
\begin{minted}{bash}
gcc hello-world.c 
./a.out 
\end{minted}

The process is extremely similar when running C++ code, as we should see below. The first step is to create the file:
\begin{minted}{bash}
touch hello-world.c
\end{minted}

Write the code inside it:
\begin{minted}{c}
#include <iostream>

using namespace std;

int main(){
    cout << "hello world from cpp" << endl;
    return 0;
}
\end{minted}

Run it via command line:
\begin{minted}{bash}
g++ hello-world.cpp
./a.out 
\end{minted}

% -----------------------------------------------------------
% -----|-----|-----|-----|-----|-----|-----|-----|-----|-----
% -----------------------------------------------------------

\section{\LaTeX}
I shall take notes with \LaTeX{} and then convert them to Markdown. To convert it into my README.md, I'll just use the following:
\begin{minted}{bash}
pandoc main.tex -o README.md --from=latex --to=gfm --standalone
\end{minted}

\bibliographystyle{plain}
\bibliography{references}
% \bibliographystyle{alpha}
% \bibliography{sample}

\end{document}