\documentclass{article}

% Language setting
% Replace `english' with e.g. `spanish' to change the document language
\usepackage[english]{babel}

% Set page size and margins
% Replace `letterpaper' with `a4paper' for UK/EU standard size
\usepackage[letterpaper,top=2cm,bottom=2cm,left=3cm,right=3cm,marginparwidth=1.75cm]{geometry}

% Useful packages
\usepackage{amsmath}
\usepackage{graphicx}
\usepackage[colorlinks=true, allcolors=blue]{hyperref}

% quotes
\usepackage{dirtytalk}

% code blocks
\usepackage[outputdir=build, cachedir=build/_minted-notes]{minted}

\title{Beginner at C/C++}
\author{Camilo de Lellis}

\begin{document}
\maketitle
\tableofcontents

\section{Introduction to C++}

\subsection{What is C++}
% TO-DO: tell the historical context for the conceiving of C++
C++ is a object oriented programming language being used in the most diverse applications. For this reason it is called a \textbf{general purpose language}. It is a superset of C, being a extension including object oriented features that C lacks.

\subsection{Why use C++}
% TO-DO: explain technically why people choose C++
I've taken a liking to C/C++ because of the career path I intend to take. I want to work with GPU development, with the development of software much more closely to hardware. I like machine learning and want to work with algorithm optimization aswell. We'll see how this will turn out. I intend to write these notes with C++ in mind because it seems the more suitable from a market perspective, but I want to compare the code to C as much as possible and say where it would be better to implement something in C and when it would be better to do so in C++.

\section{How to run C code}
Running C code is a fairly simple process. We shall now see the steps needed for doing so.

\subsection{Create the file}
\begin{minted}{bash}
    touch hello-world.c
\end{minted}

\subsection{Write the code inside it}
\begin{minted}{c}
    #include <stdio.h>

    int main( ) {
        printf("hello, world");
        return 0;
    }
\end{minted}

\subsection{Run it via command line}
\begin{minted}{bash}
    gcc hello-world.c 
    ./a.out 
\end{minted}

\section{How to run C++ code}
The process is extremely similar to running C code, as we should see below.

\subsection{Create the file}
\begin{minted}{bash}
    touch hello-world.c
\end{minted}

\subsection{Write the code inside it}
\begin{minted}{c}
    #include <iostream>

    using namespace std;

    int main(){
        cout << "hello world from cpp" << endl;
        return 0;
    }
\end{minted}

\subsection{Run it via command line}
\begin{minted}{bash}
    g++ hello-world.cpp
    ./a.out 
\end{minted}

\section{Rewriting this in \LaTeX}
I will now take notes using \LaTeX{} and then convert them to Markdown. Creating the file:

\begin{minted}{bash}
    touch main.tex
\end{minted}

After that, I'll write everything that will be available through this repo's version control.

To convert it into my README.md again, I'll just use the following:
\begin{minted}{bash}
    pandoc main.tex -o README.md --from=latex --to=gfm --standalone
\end{minted}

\bibliographystyle{alpha}
\bibliography{sample}

\end{document}